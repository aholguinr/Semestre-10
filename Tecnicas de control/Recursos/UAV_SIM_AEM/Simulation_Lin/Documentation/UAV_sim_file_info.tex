\title{UAV Simulation File Information}
\author{Austin Murch}
\documentclass[12pt]{article}
\usepackage{fullpage}
\begin{document}
\maketitle
\tableofcontents

\section{Introduction}

This document is a collection of the embedded README blocks and m-file help comments for the UMN UAV simulation, developed by the UAV Research Group at the University of Minnesota. The UAV simulation model is written in the Matlab/Simulink environment using the Aerospace Blockset.  Three simulation environments are maintained: a basic nonlinear simulation, a Software-In-the-Loop simulation, and a Processor-In-the-Loop simulation. All three simulations share the same plant dynamics, actuator, sensor, and environmental models via Simulink Libraries. Aircraft and environmental parameters are set in m-files and shared between the simulations. Two aircraft models are maintained, one for the Ultra Stick 25e and one for the FASER aircraft.
\subsection{MATLAB Version}

The UMN UAV simulation was developed with 32-bit MATLAB R2010a. Users have reported successfully using R2009b; however, R2010a or later is recommended. R2009a is known to fail with the SIL simulation.
\section{Nonlinear Simulation: UAV\_NL}
\subsection{M-Files}
\subsubsection{example1.m}
\begin{verbatim}
  example1.m
 
 -------------- Doublet response, NonLinear and Linear Models -------------
 
  Script trims the model to a level flight condition and linearizes.
  It compares doublet responses between full nonlinear sim and 
  the full and decoupled linearized models
 
  University of Minnesota 
  Aerospace Engineering and Mechanics 
  Copyright 2011 Regents of the University of Minnesota. 
  All rights reserved.
 
  SVN Info: $Id: example1.m 559 2011-09-01 19:44:55Z murch $


\end{verbatim}

\subsubsection{example2.m}
\begin{verbatim}
  example2.m
  
 ---------- Trim & linearize over a range of flight conditions ------------
 
  Script calculates a set of level flight trim condtions and linear models
  for different airspeeds. Plots trim conditions and dynamic mode
  characteristics as a function of airspeed.
 
  University of Minnesota 
  Aerospace Engineering and Mechanics 
  Copyright 2011 Regents of the University of Minnesota. 
  All rights reserved.
 
  SVN Info: $Id: example2.m 312 2011-04-01 16:15:59Z murch $


\end{verbatim}

\subsubsection{linear\_aero.m}
\begin{verbatim}
 #eml
  This function uses the linear derivatives to compute the 6 aerodynamic coefficients


\end{verbatim}

\subsubsection{linearize\_tutorial.m}
\begin{verbatim}
  UMN UAV Simulation: Linearize Tutorial
  This tutorial walks through the steps of linearizing the UMN UAV
  Simulation model. Most of these steps are handled in the "setup.m" and
  "linearize_UAV.m" functions provided with the UMN UAV sim. However, this
  tutorial will give you an in-depth understanding of how these functions
  work.

    Published output in the Help browser
       showdemo linearize_tutorial


\end{verbatim}

\subsubsection{make\_faser\_aero\_symmetric.m}
\begin{verbatim}
  This script makes the FASER baseline aerodynamic data symmetric, and
  saves the offsets in a new data structure so they can be added in if
  desired.


\end{verbatim}

\subsubsection{model\_check.m}
\begin{verbatim}
  model_check.m
  UAV_NL Model Verification
 
  Compares the linear/nonlinear doublet response of the current simulation 
  model (blue/green lines) with the checkcase data (red/black).
 
  University of Minnesota 
  Aerospace Engineering and Mechanics 
  Copyright 2011 Regents of the University of Minnesota. 
  All rights reserved.
 
  SVN Info: $Id: model_check.m 314 2011-04-05 16:53:30Z murch $


\end{verbatim}

\subsubsection{setup.m}
\begin{verbatim}
  setup.m
  UAV Nonlinear Simulation setup
 
  This script will setup the nonlinear simulation (UAV_NL.mdl) and call
  trim and linearization routines. Select the desired aircraft here in this
  script, via the "UAV_config()" function call.
 
  Note: the UAV_NL.mdl model is not opened by default. This is not
  necessary to trim, linearize, and simulate via command line inputs.
 
  Calls: UAV_config.m
         trim_UAV.m
         linearize_UAV.m
        
  University of Minnesota 
  Aerospace Engineering and Mechanics 
  Copyright 2011 Regents of the University of Minnesota. 
  All rights reserved.
 
  SVN Info: $Id: setup.m 720 2011-11-23 18:30:24Z murch $


\end{verbatim}

\subsubsection{trim\_tutorial.m}
\begin{verbatim}
  UMN UAV Simulation: Trim Tutorial
  This tutorial walks through the steps of trimming the UMN UAV Simulation
  model. Most of these steps are handled in the "setup.m" and "trim_UAV.m"
  functions provided with the UMN UAV sim. However, this tutorial will give
  you an in-depth understanding of how these functions work.

    Published output in the Help browser
       showdemo trim_tutorial


\end{verbatim}

\subsection{Simulink Blocks}
\subsubsection{UAV\_NL}
\begin{verbatim}
Nonlinear UAV Simulation

The nonlinear simulation has the Nonlinear UAV Model only (no actuators
or sensor models).  Top level inputs and outputs are used for generating
and storing trim conditions and linear models. The trim condition
generated with this model is used for the other simulations.  The
aircraft configuration, trim condition, and linear models are stored in
the Libraries directory. 

Notes:

The model automatically sets the wind and magnetic models to a 
"Bypass" option. This is done using the model's InitFcn callback. To view
or edit this function, use the Model Explorer -> UAV_NL, Callbacks tab, 
then go to "InitFcn".

The UAV_NL.mdl model is not opened by default. This is not necessary to 
trim, linearize, and simulate via command line inputs.

Light blue blocks are a UMN library link; orange blocks are a Simulink
Aerospace Blockset library link. README blocks are green.

The root level inport/outport blocks are require for trimming the model. DO
NOT delete or rename these blocks.

Control Inputs Sign Convention
Elevator: +TED
Rudder: +TEL
Aileron: +TED, da = (da_R-da_L)/2
Flap: +TED
Throttle: always positive

\end{verbatim}

\subsubsection{UAV\_NL/Nonlinear UAV Model}
\begin{verbatim}
Nonlinear UAV Model

This block implements the nonlinear UAV dynamics model. Beginning on the 
left, the Forces and Moments block models all of the relevant external
forces and moments acting on the aircraft. 

The 6DoF EOM block implements the six degree of freedom, fixed mass,
flat, non-rotating Earth, rigid body equations of motion. This block has
been modified from the original Aerospace Blockset implementation. The
inertial position vector (Xe) is no longer computed. The original block
did not have a way to add steady state winds to the inertial velocities
(Ve) prior to integrating to get Xe. Therefore, this step is performed in
the Auxiliary Equations block.

The Auxiliary Equations computes other relevant variables (such as angle
of attack, indicated airspeed, etc) from state data. In this block are
the Navigation equations.

Finally, the Environment block has Aerospace Blockset models for Earth's
atmosphere, gravity, and magnetic fields. Winds are modeled in two
portions: steady and unsteady. The steady portion is defined by a speed
and direction, and is horizontal only. The unsteady portion is made up of
a wind shear model and a turbulence model.

\end{verbatim}

\subsubsection{UAV\_NL/Nonlinear UAV Model/Environment}
\begin{verbatim}
Environment Model

This block uses the Aerospace Blockset models for Earth's atmosphere,
gravity, magnetic field, and wind. 

The atmosphere is modeled using the 1976 Standard Atmosphere block.

Winds are modeled in two portions: steady and unsteady. The steady
portion is defined by a speed and direction, and is horizontal only. The
unsteady portion is made up of a wind gust model and a turbulence model. 

The Dryden Wind Turbulence Model is used to model turbulence. The
intensity of the turbulence at low altitude (<1000ft) is determined by
the wind speed and direction; this is set as separate variables from the
steady wind speed and direction. The turbulence is off by default, and
can be enabled by setting the Env.Winds.TurbulenceOn boolean in
UAV_config.m.

The Discrete Gust Model is used to model wind gusts. Parameters are time
on, duration (length), and amplitude, in three axes (u,v,w). These
parameters are set in Env.Winds, in UAV_config.m

Note the booleans in the Env.Winds stucture to turn on each wind
component: >> Env.Winds

ans = 

     TurbulenceOn: 0
    TurbWindSpeed: 0
      TurbWindDir: 0
           GustOn: 0
    GustStartTime: 0
       GustLength: [1 1 1]
    GustAmplitude: [1 1 1]
     SteadyWindOn: 0
        WindSpeed: 0
          WindDir: 0


The WGS84 Gravity Model is used to model Earth's gravity as a function of
latitude, longitude, and altitude.

The World Magnetic Model 2005 is used to model Earth's magnetic field as
a function of latitude, longitude, and altitude. The current decimal year
is input.

*Note for trim/linearizing: the Dryden turbulence model and the World
Magnetic Model 2005 have many internal states, which makes trimming and
linearizing difficult. Thus "Bypass" blocks are substituted for the Winds
and Magnetic Model blocks using Configurable Subsystems. For the
UAV_NL.mdl, the block choices are automatically set to "Bypass" using the
models "InitFcn" callback, which can be viewed by using the Model
Explorer, selecting UAV_NL, and going to the "Callbacks" tab. The Nonlinear
UAV Model block will then show up as a Parameterized Link (red arrow in the
lower left corner). The SIL and PIL sims use the default blocks, and should
have a normal library link (black arrow).




\end{verbatim}

\subsubsection{UAV\_NL/Nonlinear UAV Model/Forces and Moments}
\begin{verbatim}
Forces and Moments

This block models all of the relevant external forces and
moments acting on the aircraft. The Aerodynamic Forces and Moments block
contains the aero models, and is a masked subsystem. The input parameter to
this block is a boolean which controls which aero model (either FASER or
the Ultrastick) is used.

The Gravitational Force block models the effect of Earth's graviy field on 
the aircraft.

The Electric Propulsion Forces and Moments block contains the electric
motor model.

Note the non-gravitational forces (nonGravForces) are output- this is what
an accelerometer on the aircraft would measure, and is used in the sensor
models.



\end{verbatim}

\subsubsection{UAV\_NL/Nonlinear UAV Model/Auxiliary Equations}
\begin{verbatim}
Auxiliary Equations

This block computes other relevant variables (such as angle of attack,
indicated airspeed, etc) from state data. Part of this block is simply to
assign signals name and create the States bus.

Working from the top of the block downwards:

The inertial velocity, V_e, is computed by using the Direction Cosine
Matrix (DCM) to transform the body-axis velocities (u,v,w) to the
inertial frame. The steady state winds are then added, and the result is
V_e. This is integrated to obtain the inertial position, X_e.  The
initial condition of the X_e integrator is set by
TrimCondition.InertialIni. V_e and X_e are inputs to the Navigation
block- see that blocks README for details.

Euler angles are bounded by +\- [pi/2, pi, 2*pi] respectively. The time 
derivatives of the Euler angles are computed using the body-axis rates
and the Euler angles.

The DCM is included in the States bus as "R_be [3x3]".

Body axis velocites and rates, and their derivatives are included. The 
unsteady (turbulence and gusts) winds are added to the body axis
velocities and rates.

Inertial accelerations are computed by transforming the body-axis
accelerations with the DCM. This step is simplified since we are using a
non-rotating Earth.

WindAxesParam is the true airspeed, angle of attack, and sideslip angle.
The derivatives of alpha and beta are computed using a deriative block.
Mach number is simply the ratio between true airspeed and the speed of
sound.

Accels [m/s^2] is what an accelerometer would read onboard the aircraft.

*Note on winds: dividing the wind components into steady and unsteady 
components is necessary because we do not use an intermediate atmospheric
reference frame to account for the motion of the air mass relative to the
Earth. We can approximate the physical effects of wind without an
additional reference frame by splitting the wind into steady and unsteady
components, where the steady component is added to the inertial
velocities and the unsteady component is added to the body-axis
velocities allows us to 

\end{verbatim}

\subsubsection{UAV\_NL/Nonlinear UAV Model/Auxiliary Equations/Navigation}
\begin{verbatim}
Navigation

This block is library link that contains the navigational model and
equations. Included are equations relating the flat Earth position to
latitude/longitude, a simplified 2D table lookup version of the EGM-96
Geoid model for computation of MSL/AGL altitudes, and computation of
flight path and ground track angles. 

\end{verbatim}

\subsubsection{UAV\_NL/Nonlinear UAV Model/Forces and Moments/Electric Propulsion Forces and Moments}
\begin{verbatim}
Electric Propulsion Forces and Moments

This block models the electric motor, propeller, and the resulting forces
and moments. 

The electric motor is modeled by using a table lookup to relate throttle
position to power output in Watts; power is coverted to torque by
dividing by the current motor angular velocity, omega (rad/s). This
torque is then summed with the required torque from the propeller. The
resulting net torque is divided by the combined motor/propeller inertia,
yielding omega_dot, which is integrated to get the current motor speed.
The initial condition of the Engine speed integrator is set in the
TrimCondition data structure.

The propeller is modelled using lookup tables of Thrust and Power
Coefficients, CT and CP, as a function of advance ratio J. These are
defined as:

CT = Thrust / (R^4 * omega^2 * 4/pi^2 * rho) 
CP = Power  / (R^5 * omega^3 * 4/pi^3 * rho) 
J  = V * pi / (omega * R)

where R is the propeller radius, omega is the angular velocity in
radians per second, and rho is the density of air. Note that the
torque coefficient for the propeller can be calculated from CP by
multiplying by n; thus the torque coefficient is not explicitly modeled.

The total forces due to the propeller is simply the thrust, which is 
not directly aligned with the body axes, so it must be rotated. These 
angles are set in the aircraft configuration m-file.

The moments due to the propeller are due to the derivative of the angular
momemtum (ie gyroscopic moments) and moments due to the position of the
thrustline relative to the center of gravity. In addition, the torque from
the motor appears as an applied moment in the equations of motion. The 
moments due to the propeller are as follows:

Mp = d/dt(Jmp*omega) + M_motor

where Jmp is the moment of inertia of the rotating portion of the motor
and propeller. Taking the derivative in the body frame, which is
non-inertial, results in:

Mp = Jmp*omega_dot + [p;q;r] X omega*Jmp + M_motor

where [p;q;r] is the body axis angular velocity. Since the
rotation axis is NOT aligned with the body x-axis, we must rotate these the
angular moment terms by a rotation matrix (L_mb) before computing the cross 
product:

Mp = L_mb*(Jmp*omega_dot + M_motor) + [p;q;r] X L_mb*omega*Jmp 

The data for the lookup tables and all of the needed aircraft parameters
are stored in the aircraft configuration data structure (AC).

>> AC.Prop

ans = 

                     CT: [-0.4314 1.0800 -0.8960 0.1089 0.0604]
                     CP: [0.5054 -0.5304 0.0412 0.0166 0.0223]
                 Radius: 0.1524
                  Power: [174.4600 70.1350 -4.3900]
    ThrottleOutputLimit: [1x1 struct]
        OmegaSaturation: [1x1 struct]
                    Jmp: 1.2991e-004
                 Angles: [0 0 0.0524]

\end{verbatim}

\subsubsection{UAV\_NL/Nonlinear UAV Model/Forces and Moments/Aerodynamic Forces and Moments/Aero Model/Ultrastick}
\begin{verbatim}
Ultrastick Aerodynamic Model

This block models the Ultrastick aerodynamics using a linear derivatives.
The force coefficients are in the wind axes, but the moment coefficients 
are in body axes, so the six coefficients are CL, CDw, CYw, Cl, Cm, Cn. 
See Klein, Morelli, Aircraft System Identification, pg 41. Thus in the 
Aero lib, the force and moment transformation block must be set as though
outgoing moment coefficients are in wind axes so no transformation is 
applied.

The derivatives are stored in the aircraft configuration data structure
(AC). They are loaded into AC.Aero when 'Ulstrastick' is selected with
UAV_config.m.  The derivatives themselves are stored in 
Ultrastick_config.m. An example is the derivatives for CL:

>>  AC.Aero.CL

ans = 

        zero: 0.1086
       alpha: 4.5800
       dflap: 0.7400
       delev: 0.0983
    alphadot: 1.9724
           q: 6.1639
        minD: 0.2300

The majority of the derivatives have been indentified from flight test
data. The drag model was derived from first principles.

\end{verbatim}

\section{SIL Simulation: UAV\_SIL}
\subsection{M-Files}
\subsubsection{SIL\_montecarlo.m}
\begin{verbatim}
  SIL Simulation Monte Carlo Setup
   The commands in this script will set conditions for the SIL simulation
   for a range of environmental conditions and model uncertainties, run and
   save the simulation data, and compare the results.
 
  University of Minnesota 
  Aerospace Engineering and Mechanics 
  Copyright 2011 Regents of the University of Minnesota. 
  All rights reserved.
 
  SVN Info: $Id: SIL_montecarlo.m 337 2011-04-15 15:20:27Z murch $


\end{verbatim}

\subsubsection{compare\_sim2flight.m}
\begin{verbatim}
  compare_sim2flight.m
 
  This script will compare flight data to SIL simulation data. Change the
  flight data file and simulation data file in the section below titled
  "User Input".
 
  University of Minnesota
  Aerospace Engineering and Mechanics
  Copyright 2011 Regents of the University of Minnesota.
  All rights reserved.
 
  SVN Info: $Id: compare_sim2flight.m 743 2011-12-09 19:19:12Z murch $


\end{verbatim}

\subsubsection{mkosswp.m}
\begin{verbatim}


\end{verbatim}

\subsubsection{model\_check.m}
\begin{verbatim}
  model_check.m
  UAV_SIL Model Verification
 
  This script runs "plot_and_save.m", which plots the pitch and roll angle
  doublet response using the current simulation model and controller. These
  results are compared to a stored simulation run of the same inputs using
  the baseline controller. Users can use this script to evaluate and
  compare the performance of their controller or model relative to a flight
  tested baseline.
 
  University of Minnesota 
  Aerospace Engineering and Mechanics 
  Copyright 2011 Regents of the University of Minnesota. 
  All rights reserved.
 
  SVN Info: $Id: model_check.m 337 2011-04-15 15:20:27Z murch $


\end{verbatim}

\subsubsection{plot\_SIL.m}
\begin{verbatim}
  plot_SIL.m
  UAV_SIL sim plot and comparison tool
 
  Input file names of saved simulation results (simData structure) and this
  function will co-plot the results. If no file name is input, the file
  "simData.mat" will be used.
 
  University of Minnesota
  Aerospace Engineering and Mechanics
  Copyright 2011 Regents of the University of Minnesota.
  All rights reserved.
 
  SVN Info: $Id: plot_SIL.m 559 2011-09-01 19:44:55Z murch $


\end{verbatim}

\subsubsection{setup.m}
\begin{verbatim}
  setup.m
  UAV Software-in-the-Loop Simulation setup
 
  This script will setup the SIL simulation. Stored aircraf configuration
  and trim conditions are used.
 
  University of Minnesota
  Aerospace Engineering and Mechanics
  Copyright 2011 Regents of the University of Minnesota.
  All rights reserved.
 
  SVN Info: $Id: setup.m 725 2011-11-23 21:29:55Z murch $


\end{verbatim}

\subsubsection{simulate\_and\_save.m}
\begin{verbatim}
  simulate_and_save.m
  UAV Software-in-the-Loop simulate_and_save function
 
  This function runs the SIL sim and saves the results to a file with the
  given name. If no file name is given, "simData" is used. A simulation
  time can also be input; the default is 45 seconds.
 
  University of Minnesota 
  Aerospace Engineering and Mechanics 
  Copyright 2011 Regents of the University of Minnesota. 
  All rights reserved.
 
  SVN Info: $Id: simulate_and_save.m 638 2011-09-30 20:27:53Z murch $


\end{verbatim}

\subsubsection{write\_sysid\_signal\_header.m}
\begin{verbatim}


\end{verbatim}

\subsection{Simulink Blocks}
\subsubsection{UAV\_SIL}
\begin{verbatim}
UAV Software-in-the-Loop Simulation

This Simulink model contains a nonlinear UAV model with closed-loop
feedback control provided by a mex-function written in C. Actuator
dynamics and sensor noise are modeled, and the simulation data is 
exported to the workspace via the "Flight Data Display" block.  

Light blue blocks are a UMN library link; orange blocks are a Simulink
Aerospace Blockset library link. README blocks are green.

1) Make sure that you have a C compiler installed that can interface with
MATLAB. You can setup the default lcc compiler of MATLAB by typing "mex
-setup"

2) Run the file 'setup.m'. The default trim condition will be used.
Change the string variable "control_code_path" to specify the path and name
of the controller source code, or specify a different Variant with the
controller_mode variable.

3) Run the file 'plot_SIL' which will generate plots of the simulation
response. Run 'model_check.m' to compare the current sim response to
stored checkcase data. If the file runs successfully and the plots
corresponded with the provided results, the system should be working
properly.  If you encounter problems with this setup, please contact
descobar@aem.umn.edu.

4) Run the file 'SIL_montecarlo.m' to run and save simulation runs with 
varying enviromental conditions and aerodynamic model parameters.

5) Run the file 'compare_sim2flight.m' to compare flight data to SIL 
simulation data.

You can change the trim conditon without changing directories to NL_Sim;
simply call trim_UAV normally. Note that the UAV_NL model will be loaded
invisibly and will use your current workspace variables.



\end{verbatim}

\subsubsection{UAV\_SIL/Control Software}
\begin{verbatim}
Control Law Block

This block uses Model Referencing to allow the user to easily switch out
control law implementations. An example usage would be first a user develops
a control law using simulink blocks. This controller is referred to as the
"simulink controller" and would be selected by setting the variable 
"controller_mode" to 2. Once the development is completed, the user then 
implements the Simulink controller into C-code suitable for integration 
with the UAV software. This controller is referred to as the "flight code"
and would be selected by setting "controller_mode" to 1. 

This is done using the Simulink.Variant object. See the documentation for 
more detail. The user can specify any number of variants for the Control
Software block; edit the ModelReferenceParameters by right-clicking this
block to set which Simulink model is reference for each Variant object. 
Note that these objects must be present in the base workspace. It
is strongly recommended that you use the "simulink_controller.mdl" file as
a starting place and "Save As" to a different file name.

In summary, to switch the Control Software, modify the "contoller_mode" 
variable:
1 = flight code controller (C implementation) 
2 = simulink controller (empty simulink model)


See "Software\Documentation\UAV_controllaw_ICD.pdf" for details on the 
input/output signals.

The control_cmd signal must have the following order:
throttle
elevator
rudder
l_aileron
r_aileron
l_flap
r_flap

Control Inputs Sign Convention
TED = Trailing Edge Down
TEL = Trailing Edge Left
------------------------
Elevator: +TED
Rudder: +TEL
Aileron: +TED, da = (da_R-da_L)/2
Throttle: always positive
Flap: +TED


The refrence command signal must have the following order:
phi_ref
theta_ref
Note the trimmed value of pitch angle theta is included in the reference
command. Use the Doublet Generator block (or other signal generating block)
to input reference commands to the Control Software.


\end{verbatim}

\section{PIL Simulation: UAV\_PIL}
\subsection{M-Files}
\subsubsection{plot\_pil.m}
\begin{verbatim}
  plot_pil.m
  UAV_PIL sim plot
 
  Input file names of saved simulation results (pilSimData structure) and this
  function will co-plot the results. If no file name is input, the file
  "pilSimData.mat" will be used.
 
  University of Minnesota
  Aerospace Engineering and Mechanics
  Copyright 2011 Regents of the University of Minnesota.
  All rights reserved.
 
  SVN Info: $Id: plot_pil.m 559 2011-09-01 19:44:55Z murch $


\end{verbatim}

\subsubsection{publish\_plots.m}
\begin{verbatim}
  Execute this code to publish the plots to a pdf file


\end{verbatim}

\subsubsection{save\_pil\_data.m}
\begin{verbatim}
  save_pil_data.m
   UAV Processor-in-the-Loop save_pil_data script
 
  This script saves the results to a file with the
  name specified in variable savename. 
 
  University of Minnesota 
  Aerospace Engineering and Mechanics 
  Copyright 2011 Regents of the University of Minnesota. 
  All rights reserved.
 
  SVN Info: $Id: save_pil_data.m 722 2011-11-23 19:47:03Z murch $


\end{verbatim}

\subsubsection{setup.m}
\begin{verbatim}
  setup.m
  UAV Processor-in-the-Loop Simulation setup
 
  IMPORTANT: Mathworks Real Time Windows Target is only supported for
  32-bit machines. http://www.mathworks.com/products/rtwt/requirements.html
 
  University of Minnesota
  Aerospace Engineering and Mechanics
  Copyright 2011 Regents of the University of Minnesota.
  All rights reserved.
 
  SVN Info: $Id: setup.m 722 2011-11-23 19:47:03Z murch $


\end{verbatim}

\subsection{Simulink Blocks}
\subsubsection{UAV\_PIL}
\begin{verbatim}

UAV Processor-in-the-Loop Simulation

This Simulink model contains a nonlinear UAV model with hardware interfaces
to connect to the UAV flight computer hardware. Aircraft state data is 
visualized via FlightGear and the UAV Ground Control Station software.

Light blue blocks are a UMN library link; orange blocks are a Simulink
Aerospace Blockset library link. README blocks are green. Yellow blocks
indicate an external interface.

You can change the trim conditon without changing directories to NL_Sim;
simply call trim_UAV normally. Note that the UAV_NL model will be loaded
invisibly and will use your current workspace variables.

For the first 10 seconds, trim settings are applied so the operator has a
chance to manually control the aircraft.

\end{verbatim}

\subsubsection{UAV\_PIL/To FlightGear}
\begin{verbatim}

To FlightGear

This block uses the Aerospace Blockset tools to send simulation data to
FlightGear for visualization. See the Simulink documentation for more
details these blocks. The latest FlightGear version supported is v1.9.1.
FlightGear can be running on the same computer as the PIL sim or can be on
another computer connected via LAN (note the destination IP address and
port setting must be updated in this case).

FlightGear is started automatically in the setup.m script, using 
"StartFlightGear.bat". Edit this file to set the correct path to your
FlightGear installation.

\end{verbatim}

\subsubsection{UAV\_PIL/MPC5200/Sensor Data
to MPC5200 
via Serial Port}
\begin{verbatim}

To MPC5200 (via Serial)

This block streams a binary data packet to the MPC5200B. The packet header
was chosen to be the same as the SiRF GPS Binary Protocol for familiarity.
The data values are representative of the primary sensor data onboard the
aircraft. The control mode (ie manual or auto) is also an input.

Data are sent as double precision values, and are decoded in L_pil_daq.c.

\end{verbatim}

\subsubsection{UAV\_PIL/MPC5200/Control Inputs
from MPC5200
via Serial Port}
\begin{verbatim}

Control Inputs

This block contains the serial interface to the MPC5200B that recieves the 
actuator commands. The commands are in the following order, and are sent
as double precision (8 bytes each). Units are radians.

Throttle (0-1)
Elevator
Rudder
Left Aileron
Right Aileron
Left Flap
Right Flap


\end{verbatim}

\section{Common M-Files}
\subsection{FASER\_config.m}
\begin{verbatim}
  function [AC] = FASER_config()
 
 
  FASER configuration file. Sets aircraft parameters.
  Called from: UAV_config.m
 
  University of Minnesota 
  Aerospace Engineering and Mechanics 
  Copyright 2011 Regents of the University of Minnesota. 
  All rights reserved.
 
  SVN Info: $Id: FASER_config.m 695 2011-11-15 16:17:42Z murch $


\end{verbatim}

\subsection{UAV\_config.m}
\begin{verbatim}
  function [AC,Env] = UAV_config(aircraft,savefile)
 
  Defines aircraft parameters. Input desired aircraft and savefile boolean.
  Sets Env data structure.
 
  University of Minnesota 
  Aerospace Engineering and Mechanics 
  Copyright 2011 Regents of the University of Minnesota. 
  All rights reserved.
 
  SVN Info: $Id: UAV_config.m 314 2011-04-05 16:53:30Z murch $


\end{verbatim}

\subsection{Ultrastick\_config.m}
\begin{verbatim}
  function [AC] = Ultrastick_config()
 
  Ultra Stick 25e configuration file. Sets aircraft parameters.
  Called from: UAV_config.m
 
  University of Minnesota 
  Aerospace Engineering and Mechanics 
  Copyright 2011 Regents of the University of Minnesota. All rights reserved.
 
  SVN Info: $Id: Ultrastick_config.m 695 2011-11-15 16:17:42Z murch $


\end{verbatim}

\subsection{busnames2excel.m}
\begin{verbatim}
  function busnames2excel(savename)
 
  Takes input/output bus signal names from UAV_NL.mdl and stores them in a
  Excel file. The default name for this file is 'UAV_sim_ICD.xlsx'.
 
  Output signal names are taken from the "States" and "EnvData" bus
  selectors. Input signal names are taken from the "Control Inputs" bus
  creator.
 
  University of Minnesota 
  Aerospace Engineering and Mechanics 
  Copyright 2011 Regents of the University of Minnesota. 
  All rights reserved.
 
  SVN Info: $Id: busnames2excel.m 284 2011-03-03 15:07:19Z murch $


\end{verbatim}

\subsection{eigpara.m}
\begin{verbatim}
  function [wd, T, wn, zeta] = eigpara(lambda)
 
  [wd, T, wn, zeta] = eigparam(lambda)
 
  Return the parameters of a complex eigenvalue
  Inputs:
    lambda = a complex eigenvalue
  Outputs:
    wd = the damped natural frequency
    T = the period
    wn = the natural frequency
    zeta = the damping
 
  University of Minnesota 
  Aerospace Engineering and Mechanics 
  Copyright 2011 Regents of the University of Minnesota. 
  All rights reserved.
 
  SVN Info: $Id: eigpara.m 284 2011-03-03 15:07:19Z murch $


\end{verbatim}

\subsection{linearize\_UAV.m}
\begin{verbatim}
  [longmod,spmod,latmod,linmodel]=linearize_UAV(OperatingPoint,verbose)
 
  Linearizes the UAV model about a given operating point using
  ../NL_Sim/UAV_NL.mdl. This function can be called from any of the three
  sim directories. However, this function will use your workspace
  variables. Requires the Control System Toolbox and Simulink Control
  Design.
 
  Inputs:
    OperatingPoint - Operating point object of a trim condition
    use_uvw        - boolean flag to use u,v,w as linear model outputs 
                      instead of V, alpha, beta; defaults to "false"
    verbose        - boolean flag to suppress output; default "true"
 
  Outputs:
    longmod  - longitudinal linear model
    spmod    - short period approximation
    latmod   - lateral directional linear model
    linmodel - full linear model
 
 
  University of Minnesota 
  Aerospace Engineering and Mechanics 
  Copyright 2011 Regents of the University of Minnesota. 
  All rights reserved.
 
  SVN Info: $Id: linearize_UAV.m 648 2011-10-07 17:14:56Z murch $


\end{verbatim}

\subsection{trim\_UAV.m}
\begin{verbatim}
 [TrimCondition,OperatingPoint]=trim_UAV(TrimCondition,AC,savefile,verbose)
 
  Trims the UAV simulation to target conditions using ..\NL_Sim\UAV_NL.mdl.
  This function can be called from any of the three sim directories.
  However, this function will use your workspace variables. Requires 
  Simulink Control Design.
 
  Set the trim target as shown below.
 
  Inputs:
      TrimCondition - Initial aircraft state, with a structure called
      "target", which has some subset of the following fields:
      V_s          - True airspeed (m/s)
      alpha        - Angle of attack (rad)
      beta         - Sideslip (rad), defaults to zero
      gamma        - Flight path angle (rad), defaults to zero
      phi          - roll angle (rad)
      theta        - pitch angle (rad)
      psi          - Heading angle (0-360)
      phidot       - d/dt(phi) (rad/sec), defaults to zero
      thetadot     - d/dt(theta) (rad/sec), defaults to zero
      psidot       - d/dt(psi) (rad/sec), defaults to zero
      p            - Angular velocity (rad/sec)
      q            - Angular velocity (rad/sec)
      r            - Angular velocity (rad/sec)
      h            - Altitude above ground level (AGL) (m)
      elevator     - elevator control input, rad. 
      aileron      - combined aileron control input, rad. (da_r - da_l)/2
      l_aileron    - left aileron control input, rad
      r_aileron    - right aileron control input, rad
      rudder       - rudder control input, rad. 
      throttle     - throttle control input, nd. 
      flap         - flap control input, rad. Defaults to fixed at zero.
 
      AC       - Aircraft configuration structure, from UAV_config.m
      savefile - boolean flag to save trim condition; default "true"
      verbose  - boolean flag to suppress output; default "true"
 
  Outputs:
    TrimCondition  - values of state and control surfaces at trim.
    OperatingPoint - Simulink OperatingPoint object to use with
    linearization
 
  Unspecified variables are free, or defaulted to the values shown above.
  To force a defaulted variable to be free define it with an empty matrix.
  For example, by default beta=0  but "target.beta=[];" will allow beta
  to be free in searching for a trim condition.
 
  Examples:
  TrimCondition.target = struct('V_s',17,'gamma',0); % straight and level
  TrimCondition.target = struct('V_s',17,'gamma',5/180*pi); % level climb
  TrimCondition.target = struct('V_s',17,'gamma',0,...
                      'psidot',20/180*pi); % level turn
  TrimCondition.target = struct('V_s',17,'gamma',5/180*pi,...
                      'psidot',20/180*pi); % climbing turn
  TrimCondition.target = struct('V_s',17,'gamma',0,...
                       'beta',5/180*pi); % level steady heading sideslip
 
  Based in part on the trimgtm.m script by David Cox, NASA LaRC 
  (David.E.Cox@.nasa.gov)
 
  University of Minnesota 
  Aerospace Engineering and Mechanics 
  Copyright 2011 Regents of the University of Minnesota. 
  All rights reserved.
 
  SVN Info: $Id: trim_UAV.m 684 2011-11-09 21:06:52Z murch $


\end{verbatim}

\end{document}